% Table 3.2: Theorem 3.5 Properties Summary
% Location: Chapter 3 (Theoretical Foundations)
% Purpose: Summarize key properties and guarantees of the falsifiability framework
% LaTeX packages required: booktabs, tabularx

\begin{table}[htbp]
\centering
\caption{Properties of Falsifiability Criterion (Theorem 3.5)}
\label{tab:theorem_properties}
\small
\begin{tabularx}{\textwidth}{lXp{0.25\textwidth}}
\toprule
\textbf{Property} & \textbf{Description} & \textbf{Implication} \\
\midrule
\textbf{Uniqueness} & For a given attribution method and image pair, the falsifiability test produces a deterministic verdict (modulo sampling variance in counterfactual generation) & Reproducible evaluation: same inputs always yield same verdict (within statistical confidence) \\[8pt]

\textbf{Method-Agnostic} & Framework applies to any attribution method producing importance scores $\phi \in \mathbb{R}^m$ (Grad-CAM, SHAP, LIME, IG, novel methods) & Universal evaluation protocol: no need for method-specific modifications \\[8pt]

\textbf{Computationally Tractable} & Runtime scales linearly with number of counterfactuals $K$ and quadratically with image resolution (via gradient computation) & Practical deployment: 1000 images processable in $\sim$1.1 hours (RTX 3090) \\[8pt]

\textbf{Statistical Rigor} & Verdicts include $p$-values from Mann-Whitney U tests, confidence intervals from bootstrap resampling, and effect sizes (Cohen's $d$) & Quantifiable uncertainty: verdicts are not binary but probabilistic with known error rates \\[8pt]

\textbf{Embedding-Space Native} & Test operates directly on learned face embeddings $\phi \in \mathbb{R}^{512}$ using geodesic distances on hypersphere & Geometry-aware: respects ArcFace/CosFace angular margin structure \\[8pt]

\textbf{Plausibility-Constrained} & Counterfactuals must satisfy perceptual plausibility: LPIPS $< 0.4$ and remain on face manifold & Realistic testing: avoids out-of-distribution artifacts from naive perturbations \\

\bottomrule
\end{tabularx}
\end{table}

% Alt-text (for accessibility):
% Table listing six key properties of the falsifiability criterion (Theorem 3.5).
% Properties ensure uniqueness, method-agnosticism, computational tractability,
% statistical rigor, embedding-space awareness, and plausibility constraints.
% These properties enable reproducible, universal, and realistic evaluation of attribution methods.
