% Table 2.4: Counterfactual XAI Methods Taxonomy
% Location: Chapter 2 (Literature Review)
% Purpose: Classify counterfactual explanation methods using Stepin et al. (2021) taxonomy
% LaTeX packages required: booktabs, tabularx

\begin{table}[htbp]
\centering
\caption{Counterfactual XAI Methods Taxonomy}
\label{tab:counterfactual_taxonomy}
\small
\begin{tabularx}{\textwidth}{llXp{0.25\textwidth}}
\toprule
\textbf{Reasoning Type} & \textbf{Explanation Type} & \textbf{Description} & \textbf{Example Methods / Papers} \\
\midrule
\multirow{2}{*}{\textbf{Causal}} & Contrastive & Explanation compares actual outcome with alternative outcome based on causal intervention, answers "Why P rather than Q?" & Structural Causal Models (Pearl 2009), Causal Bayesian Networks \cite{pearl2009causality} \\[6pt]
\cmidrule(lr){2-4}
& Counterfactual & Explanation shows minimal intervention to change outcome, answers "What if X were different?" & Counterfactual Explanations without Opening the Black Box \cite{wachter2017counterfactual}, FACE \cite{poyiadzi2020face} \\[10pt]

\midrule

\multirow{2}{*}{\textbf{Non-Causal}} & Contrastive & Explanation highlights discriminative features between classes without causal semantics & Contrastive Explanations Method (CEM) \cite{dhurandhar2018cem}, Pertinent Positives/Negatives \\[6pt]
\cmidrule(lr){2-4}
& Counterfactual & Explanation generates plausible alternatives without causal guarantees & DiCE \cite{mothilal2020dice}, GAN-based counterfactuals, This Work (plausibility-preserving perturbations) \\

\bottomrule
\end{tabularx}
\end{table}

% Alt-text (for accessibility):
% Table showing 2×2 taxonomy of counterfactual/contrastive explanation methods.
% Causal reasoning methods require causal models (Pearl's SCMs, Wachter's counterfactuals).
% Non-causal methods use discriminative features or generative models without causal guarantees.
% This work falls in non-causal counterfactual category (plausibility-preserving perturbations).
