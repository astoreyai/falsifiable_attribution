% Table 2.2: Evaluation Paradigms Comparison
% Compares plausibility, faithfulness, and falsifiability evaluation approaches
% Location: Chapter 2, Literature Review
% Alt-text: Comparison table showing three XAI evaluation paradigms (Plausibility, Faithfulness, Falsifiability)
% across seven dimensions: evaluation approach, what is measured, testable predictions, ground truth required,
% example metrics, Daubert standard compliance, and key papers

\begin{table}[htbp]
\centering
\small
\caption{Comparison of XAI Evaluation Paradigms: Plausibility, Faithfulness, and Falsifiability. This dissertation introduces the falsifiability paradigm (third row) as a novel contribution that addresses critical gaps in existing evaluation approaches.}
\label{tab:evaluation_paradigms}
\begin{tabular}{p{2.2cm}p{2.8cm}p{1.3cm}p{1.8cm}p{3cm}p{1.4cm}p{2.2cm}}
\toprule
\textbf{Evaluation Approach} & \textbf{Measures} & \textbf{Testable Prediction?} & \textbf{Ground Truth Required?} & \textbf{Example Metrics} & \textbf{Daubert Standard?} & \textbf{Key Papers} \\
\midrule
\textbf{Plausibility}
& Subjective interpretability, alignment with human intuition
& No
& No
& Human ratings (``looks reasonable''), attention visualization
& Fails
& Attention mechanisms, visualization studies \\
\midrule
\textbf{Faithfulness}
& Correlation with model behavior under perturbation
& No (correlation only)
& Partial (perturbation effects)
& Deletion AUC, insertion AUC, AOPC, infidelity
& Partial
& Adebayo et al. 2018, Kindermans et al. 2019, Tomsett et al. 2020 \\
\midrule
\textbf{Falsifiability} \newline \textit{(This work)}
& Counterfactual predictions: high-attribution features cause large $\Delta$-score, low-attribution features cause small $\Delta$-score
& Yes
& Yes (realized $\Delta$-score from counterfactuals)
& Correlation coefficient $\rho$, p-value, separation margin $\epsilon$
& Passes
& This dissertation (novel framework) \\
\bottomrule
\end{tabular}
\end{table}

% Usage notes:
% - Include in chapter_02_literature_review.tex with: % Table 2.2: Evaluation Paradigms Comparison
% Compares plausibility, faithfulness, and falsifiability evaluation approaches
% Location: Chapter 2, Literature Review
% Alt-text: Comparison table showing three XAI evaluation paradigms (Plausibility, Faithfulness, Falsifiability)
% across seven dimensions: evaluation approach, what is measured, testable predictions, ground truth required,
% example metrics, Daubert standard compliance, and key papers

\begin{table}[htbp]
\centering
\small
\caption{Comparison of XAI Evaluation Paradigms: Plausibility, Faithfulness, and Falsifiability. This dissertation introduces the falsifiability paradigm (third row) as a novel contribution that addresses critical gaps in existing evaluation approaches.}
\label{tab:evaluation_paradigms}
\begin{tabular}{p{2.2cm}p{2.8cm}p{1.3cm}p{1.8cm}p{3cm}p{1.4cm}p{2.2cm}}
\toprule
\textbf{Evaluation Approach} & \textbf{Measures} & \textbf{Testable Prediction?} & \textbf{Ground Truth Required?} & \textbf{Example Metrics} & \textbf{Daubert Standard?} & \textbf{Key Papers} \\
\midrule
\textbf{Plausibility}
& Subjective interpretability, alignment with human intuition
& No
& No
& Human ratings (``looks reasonable''), attention visualization
& Fails
& Attention mechanisms, visualization studies \\
\midrule
\textbf{Faithfulness}
& Correlation with model behavior under perturbation
& No (correlation only)
& Partial (perturbation effects)
& Deletion AUC, insertion AUC, AOPC, infidelity
& Partial
& Adebayo et al. 2018, Kindermans et al. 2019, Tomsett et al. 2020 \\
\midrule
\textbf{Falsifiability} \newline \textit{(This work)}
& Counterfactual predictions: high-attribution features cause large $\Delta$-score, low-attribution features cause small $\Delta$-score
& Yes
& Yes (realized $\Delta$-score from counterfactuals)
& Correlation coefficient $\rho$, p-value, separation margin $\epsilon$
& Passes
& This dissertation (novel framework) \\
\bottomrule
\end{tabular}
\end{table}

% Usage notes:
% - Include in chapter_02_literature_review.tex with: % Table 2.2: Evaluation Paradigms Comparison
% Compares plausibility, faithfulness, and falsifiability evaluation approaches
% Location: Chapter 2, Literature Review
% Alt-text: Comparison table showing three XAI evaluation paradigms (Plausibility, Faithfulness, Falsifiability)
% across seven dimensions: evaluation approach, what is measured, testable predictions, ground truth required,
% example metrics, Daubert standard compliance, and key papers

\begin{table}[htbp]
\centering
\small
\caption{Comparison of XAI Evaluation Paradigms: Plausibility, Faithfulness, and Falsifiability. This dissertation introduces the falsifiability paradigm (third row) as a novel contribution that addresses critical gaps in existing evaluation approaches.}
\label{tab:evaluation_paradigms}
\begin{tabular}{p{2.2cm}p{2.8cm}p{1.3cm}p{1.8cm}p{3cm}p{1.4cm}p{2.2cm}}
\toprule
\textbf{Evaluation Approach} & \textbf{Measures} & \textbf{Testable Prediction?} & \textbf{Ground Truth Required?} & \textbf{Example Metrics} & \textbf{Daubert Standard?} & \textbf{Key Papers} \\
\midrule
\textbf{Plausibility}
& Subjective interpretability, alignment with human intuition
& No
& No
& Human ratings (``looks reasonable''), attention visualization
& Fails
& Attention mechanisms, visualization studies \\
\midrule
\textbf{Faithfulness}
& Correlation with model behavior under perturbation
& No (correlation only)
& Partial (perturbation effects)
& Deletion AUC, insertion AUC, AOPC, infidelity
& Partial
& Adebayo et al. 2018, Kindermans et al. 2019, Tomsett et al. 2020 \\
\midrule
\textbf{Falsifiability} \newline \textit{(This work)}
& Counterfactual predictions: high-attribution features cause large $\Delta$-score, low-attribution features cause small $\Delta$-score
& Yes
& Yes (realized $\Delta$-score from counterfactuals)
& Correlation coefficient $\rho$, p-value, separation margin $\epsilon$
& Passes
& This dissertation (novel framework) \\
\bottomrule
\end{tabular}
\end{table}

% Usage notes:
% - Include in chapter_02_literature_review.tex with: % Table 2.2: Evaluation Paradigms Comparison
% Compares plausibility, faithfulness, and falsifiability evaluation approaches
% Location: Chapter 2, Literature Review
% Alt-text: Comparison table showing three XAI evaluation paradigms (Plausibility, Faithfulness, Falsifiability)
% across seven dimensions: evaluation approach, what is measured, testable predictions, ground truth required,
% example metrics, Daubert standard compliance, and key papers

\begin{table}[htbp]
\centering
\small
\caption{Comparison of XAI Evaluation Paradigms: Plausibility, Faithfulness, and Falsifiability. This dissertation introduces the falsifiability paradigm (third row) as a novel contribution that addresses critical gaps in existing evaluation approaches.}
\label{tab:evaluation_paradigms}
\begin{tabular}{p{2.2cm}p{2.8cm}p{1.3cm}p{1.8cm}p{3cm}p{1.4cm}p{2.2cm}}
\toprule
\textbf{Evaluation Approach} & \textbf{Measures} & \textbf{Testable Prediction?} & \textbf{Ground Truth Required?} & \textbf{Example Metrics} & \textbf{Daubert Standard?} & \textbf{Key Papers} \\
\midrule
\textbf{Plausibility}
& Subjective interpretability, alignment with human intuition
& No
& No
& Human ratings (``looks reasonable''), attention visualization
& Fails
& Attention mechanisms, visualization studies \\
\midrule
\textbf{Faithfulness}
& Correlation with model behavior under perturbation
& No (correlation only)
& Partial (perturbation effects)
& Deletion AUC, insertion AUC, AOPC, infidelity
& Partial
& Adebayo et al. 2018, Kindermans et al. 2019, Tomsett et al. 2020 \\
\midrule
\textbf{Falsifiability} \newline \textit{(This work)}
& Counterfactual predictions: high-attribution features cause large $\Delta$-score, low-attribution features cause small $\Delta$-score
& Yes
& Yes (realized $\Delta$-score from counterfactuals)
& Correlation coefficient $\rho$, p-value, separation margin $\epsilon$
& Passes
& This dissertation (novel framework) \\
\bottomrule
\end{tabular}
\end{table}

% Usage notes:
% - Include in chapter_02_literature_review.tex with: \input{../tables/chapter_02_literature/table_2_2_evaluation_paradigms}
% - Or use full path: \input{tables/chapter_02_literature/table_2_2_evaluation_paradigms}
% - Requires: \usepackage{booktabs} in preamble
% - Landscape orientation not needed (fits on portrait page)

% - Or use full path: % Table 2.2: Evaluation Paradigms Comparison
% Compares plausibility, faithfulness, and falsifiability evaluation approaches
% Location: Chapter 2, Literature Review
% Alt-text: Comparison table showing three XAI evaluation paradigms (Plausibility, Faithfulness, Falsifiability)
% across seven dimensions: evaluation approach, what is measured, testable predictions, ground truth required,
% example metrics, Daubert standard compliance, and key papers

\begin{table}[htbp]
\centering
\small
\caption{Comparison of XAI Evaluation Paradigms: Plausibility, Faithfulness, and Falsifiability. This dissertation introduces the falsifiability paradigm (third row) as a novel contribution that addresses critical gaps in existing evaluation approaches.}
\label{tab:evaluation_paradigms}
\begin{tabular}{p{2.2cm}p{2.8cm}p{1.3cm}p{1.8cm}p{3cm}p{1.4cm}p{2.2cm}}
\toprule
\textbf{Evaluation Approach} & \textbf{Measures} & \textbf{Testable Prediction?} & \textbf{Ground Truth Required?} & \textbf{Example Metrics} & \textbf{Daubert Standard?} & \textbf{Key Papers} \\
\midrule
\textbf{Plausibility}
& Subjective interpretability, alignment with human intuition
& No
& No
& Human ratings (``looks reasonable''), attention visualization
& Fails
& Attention mechanisms, visualization studies \\
\midrule
\textbf{Faithfulness}
& Correlation with model behavior under perturbation
& No (correlation only)
& Partial (perturbation effects)
& Deletion AUC, insertion AUC, AOPC, infidelity
& Partial
& Adebayo et al. 2018, Kindermans et al. 2019, Tomsett et al. 2020 \\
\midrule
\textbf{Falsifiability} \newline \textit{(This work)}
& Counterfactual predictions: high-attribution features cause large $\Delta$-score, low-attribution features cause small $\Delta$-score
& Yes
& Yes (realized $\Delta$-score from counterfactuals)
& Correlation coefficient $\rho$, p-value, separation margin $\epsilon$
& Passes
& This dissertation (novel framework) \\
\bottomrule
\end{tabular}
\end{table}

% Usage notes:
% - Include in chapter_02_literature_review.tex with: \input{../tables/chapter_02_literature/table_2_2_evaluation_paradigms}
% - Or use full path: \input{tables/chapter_02_literature/table_2_2_evaluation_paradigms}
% - Requires: \usepackage{booktabs} in preamble
% - Landscape orientation not needed (fits on portrait page)

% - Requires: \usepackage{booktabs} in preamble
% - Landscape orientation not needed (fits on portrait page)

% - Or use full path: % Table 2.2: Evaluation Paradigms Comparison
% Compares plausibility, faithfulness, and falsifiability evaluation approaches
% Location: Chapter 2, Literature Review
% Alt-text: Comparison table showing three XAI evaluation paradigms (Plausibility, Faithfulness, Falsifiability)
% across seven dimensions: evaluation approach, what is measured, testable predictions, ground truth required,
% example metrics, Daubert standard compliance, and key papers

\begin{table}[htbp]
\centering
\small
\caption{Comparison of XAI Evaluation Paradigms: Plausibility, Faithfulness, and Falsifiability. This dissertation introduces the falsifiability paradigm (third row) as a novel contribution that addresses critical gaps in existing evaluation approaches.}
\label{tab:evaluation_paradigms}
\begin{tabular}{p{2.2cm}p{2.8cm}p{1.3cm}p{1.8cm}p{3cm}p{1.4cm}p{2.2cm}}
\toprule
\textbf{Evaluation Approach} & \textbf{Measures} & \textbf{Testable Prediction?} & \textbf{Ground Truth Required?} & \textbf{Example Metrics} & \textbf{Daubert Standard?} & \textbf{Key Papers} \\
\midrule
\textbf{Plausibility}
& Subjective interpretability, alignment with human intuition
& No
& No
& Human ratings (``looks reasonable''), attention visualization
& Fails
& Attention mechanisms, visualization studies \\
\midrule
\textbf{Faithfulness}
& Correlation with model behavior under perturbation
& No (correlation only)
& Partial (perturbation effects)
& Deletion AUC, insertion AUC, AOPC, infidelity
& Partial
& Adebayo et al. 2018, Kindermans et al. 2019, Tomsett et al. 2020 \\
\midrule
\textbf{Falsifiability} \newline \textit{(This work)}
& Counterfactual predictions: high-attribution features cause large $\Delta$-score, low-attribution features cause small $\Delta$-score
& Yes
& Yes (realized $\Delta$-score from counterfactuals)
& Correlation coefficient $\rho$, p-value, separation margin $\epsilon$
& Passes
& This dissertation (novel framework) \\
\bottomrule
\end{tabular}
\end{table}

% Usage notes:
% - Include in chapter_02_literature_review.tex with: % Table 2.2: Evaluation Paradigms Comparison
% Compares plausibility, faithfulness, and falsifiability evaluation approaches
% Location: Chapter 2, Literature Review
% Alt-text: Comparison table showing three XAI evaluation paradigms (Plausibility, Faithfulness, Falsifiability)
% across seven dimensions: evaluation approach, what is measured, testable predictions, ground truth required,
% example metrics, Daubert standard compliance, and key papers

\begin{table}[htbp]
\centering
\small
\caption{Comparison of XAI Evaluation Paradigms: Plausibility, Faithfulness, and Falsifiability. This dissertation introduces the falsifiability paradigm (third row) as a novel contribution that addresses critical gaps in existing evaluation approaches.}
\label{tab:evaluation_paradigms}
\begin{tabular}{p{2.2cm}p{2.8cm}p{1.3cm}p{1.8cm}p{3cm}p{1.4cm}p{2.2cm}}
\toprule
\textbf{Evaluation Approach} & \textbf{Measures} & \textbf{Testable Prediction?} & \textbf{Ground Truth Required?} & \textbf{Example Metrics} & \textbf{Daubert Standard?} & \textbf{Key Papers} \\
\midrule
\textbf{Plausibility}
& Subjective interpretability, alignment with human intuition
& No
& No
& Human ratings (``looks reasonable''), attention visualization
& Fails
& Attention mechanisms, visualization studies \\
\midrule
\textbf{Faithfulness}
& Correlation with model behavior under perturbation
& No (correlation only)
& Partial (perturbation effects)
& Deletion AUC, insertion AUC, AOPC, infidelity
& Partial
& Adebayo et al. 2018, Kindermans et al. 2019, Tomsett et al. 2020 \\
\midrule
\textbf{Falsifiability} \newline \textit{(This work)}
& Counterfactual predictions: high-attribution features cause large $\Delta$-score, low-attribution features cause small $\Delta$-score
& Yes
& Yes (realized $\Delta$-score from counterfactuals)
& Correlation coefficient $\rho$, p-value, separation margin $\epsilon$
& Passes
& This dissertation (novel framework) \\
\bottomrule
\end{tabular}
\end{table}

% Usage notes:
% - Include in chapter_02_literature_review.tex with: \input{../tables/chapter_02_literature/table_2_2_evaluation_paradigms}
% - Or use full path: \input{tables/chapter_02_literature/table_2_2_evaluation_paradigms}
% - Requires: \usepackage{booktabs} in preamble
% - Landscape orientation not needed (fits on portrait page)

% - Or use full path: % Table 2.2: Evaluation Paradigms Comparison
% Compares plausibility, faithfulness, and falsifiability evaluation approaches
% Location: Chapter 2, Literature Review
% Alt-text: Comparison table showing three XAI evaluation paradigms (Plausibility, Faithfulness, Falsifiability)
% across seven dimensions: evaluation approach, what is measured, testable predictions, ground truth required,
% example metrics, Daubert standard compliance, and key papers

\begin{table}[htbp]
\centering
\small
\caption{Comparison of XAI Evaluation Paradigms: Plausibility, Faithfulness, and Falsifiability. This dissertation introduces the falsifiability paradigm (third row) as a novel contribution that addresses critical gaps in existing evaluation approaches.}
\label{tab:evaluation_paradigms}
\begin{tabular}{p{2.2cm}p{2.8cm}p{1.3cm}p{1.8cm}p{3cm}p{1.4cm}p{2.2cm}}
\toprule
\textbf{Evaluation Approach} & \textbf{Measures} & \textbf{Testable Prediction?} & \textbf{Ground Truth Required?} & \textbf{Example Metrics} & \textbf{Daubert Standard?} & \textbf{Key Papers} \\
\midrule
\textbf{Plausibility}
& Subjective interpretability, alignment with human intuition
& No
& No
& Human ratings (``looks reasonable''), attention visualization
& Fails
& Attention mechanisms, visualization studies \\
\midrule
\textbf{Faithfulness}
& Correlation with model behavior under perturbation
& No (correlation only)
& Partial (perturbation effects)
& Deletion AUC, insertion AUC, AOPC, infidelity
& Partial
& Adebayo et al. 2018, Kindermans et al. 2019, Tomsett et al. 2020 \\
\midrule
\textbf{Falsifiability} \newline \textit{(This work)}
& Counterfactual predictions: high-attribution features cause large $\Delta$-score, low-attribution features cause small $\Delta$-score
& Yes
& Yes (realized $\Delta$-score from counterfactuals)
& Correlation coefficient $\rho$, p-value, separation margin $\epsilon$
& Passes
& This dissertation (novel framework) \\
\bottomrule
\end{tabular}
\end{table}

% Usage notes:
% - Include in chapter_02_literature_review.tex with: \input{../tables/chapter_02_literature/table_2_2_evaluation_paradigms}
% - Or use full path: \input{tables/chapter_02_literature/table_2_2_evaluation_paradigms}
% - Requires: \usepackage{booktabs} in preamble
% - Landscape orientation not needed (fits on portrait page)

% - Requires: \usepackage{booktabs} in preamble
% - Landscape orientation not needed (fits on portrait page)

% - Requires: \usepackage{booktabs} in preamble
% - Landscape orientation not needed (fits on portrait page)

% - Or use full path: % Table 2.2: Evaluation Paradigms Comparison
% Compares plausibility, faithfulness, and falsifiability evaluation approaches
% Location: Chapter 2, Literature Review
% Alt-text: Comparison table showing three XAI evaluation paradigms (Plausibility, Faithfulness, Falsifiability)
% across seven dimensions: evaluation approach, what is measured, testable predictions, ground truth required,
% example metrics, Daubert standard compliance, and key papers

\begin{table}[htbp]
\centering
\small
\caption{Comparison of XAI Evaluation Paradigms: Plausibility, Faithfulness, and Falsifiability. This dissertation introduces the falsifiability paradigm (third row) as a novel contribution that addresses critical gaps in existing evaluation approaches.}
\label{tab:evaluation_paradigms}
\begin{tabular}{p{2.2cm}p{2.8cm}p{1.3cm}p{1.8cm}p{3cm}p{1.4cm}p{2.2cm}}
\toprule
\textbf{Evaluation Approach} & \textbf{Measures} & \textbf{Testable Prediction?} & \textbf{Ground Truth Required?} & \textbf{Example Metrics} & \textbf{Daubert Standard?} & \textbf{Key Papers} \\
\midrule
\textbf{Plausibility}
& Subjective interpretability, alignment with human intuition
& No
& No
& Human ratings (``looks reasonable''), attention visualization
& Fails
& Attention mechanisms, visualization studies \\
\midrule
\textbf{Faithfulness}
& Correlation with model behavior under perturbation
& No (correlation only)
& Partial (perturbation effects)
& Deletion AUC, insertion AUC, AOPC, infidelity
& Partial
& Adebayo et al. 2018, Kindermans et al. 2019, Tomsett et al. 2020 \\
\midrule
\textbf{Falsifiability} \newline \textit{(This work)}
& Counterfactual predictions: high-attribution features cause large $\Delta$-score, low-attribution features cause small $\Delta$-score
& Yes
& Yes (realized $\Delta$-score from counterfactuals)
& Correlation coefficient $\rho$, p-value, separation margin $\epsilon$
& Passes
& This dissertation (novel framework) \\
\bottomrule
\end{tabular}
\end{table}

% Usage notes:
% - Include in chapter_02_literature_review.tex with: % Table 2.2: Evaluation Paradigms Comparison
% Compares plausibility, faithfulness, and falsifiability evaluation approaches
% Location: Chapter 2, Literature Review
% Alt-text: Comparison table showing three XAI evaluation paradigms (Plausibility, Faithfulness, Falsifiability)
% across seven dimensions: evaluation approach, what is measured, testable predictions, ground truth required,
% example metrics, Daubert standard compliance, and key papers

\begin{table}[htbp]
\centering
\small
\caption{Comparison of XAI Evaluation Paradigms: Plausibility, Faithfulness, and Falsifiability. This dissertation introduces the falsifiability paradigm (third row) as a novel contribution that addresses critical gaps in existing evaluation approaches.}
\label{tab:evaluation_paradigms}
\begin{tabular}{p{2.2cm}p{2.8cm}p{1.3cm}p{1.8cm}p{3cm}p{1.4cm}p{2.2cm}}
\toprule
\textbf{Evaluation Approach} & \textbf{Measures} & \textbf{Testable Prediction?} & \textbf{Ground Truth Required?} & \textbf{Example Metrics} & \textbf{Daubert Standard?} & \textbf{Key Papers} \\
\midrule
\textbf{Plausibility}
& Subjective interpretability, alignment with human intuition
& No
& No
& Human ratings (``looks reasonable''), attention visualization
& Fails
& Attention mechanisms, visualization studies \\
\midrule
\textbf{Faithfulness}
& Correlation with model behavior under perturbation
& No (correlation only)
& Partial (perturbation effects)
& Deletion AUC, insertion AUC, AOPC, infidelity
& Partial
& Adebayo et al. 2018, Kindermans et al. 2019, Tomsett et al. 2020 \\
\midrule
\textbf{Falsifiability} \newline \textit{(This work)}
& Counterfactual predictions: high-attribution features cause large $\Delta$-score, low-attribution features cause small $\Delta$-score
& Yes
& Yes (realized $\Delta$-score from counterfactuals)
& Correlation coefficient $\rho$, p-value, separation margin $\epsilon$
& Passes
& This dissertation (novel framework) \\
\bottomrule
\end{tabular}
\end{table}

% Usage notes:
% - Include in chapter_02_literature_review.tex with: % Table 2.2: Evaluation Paradigms Comparison
% Compares plausibility, faithfulness, and falsifiability evaluation approaches
% Location: Chapter 2, Literature Review
% Alt-text: Comparison table showing three XAI evaluation paradigms (Plausibility, Faithfulness, Falsifiability)
% across seven dimensions: evaluation approach, what is measured, testable predictions, ground truth required,
% example metrics, Daubert standard compliance, and key papers

\begin{table}[htbp]
\centering
\small
\caption{Comparison of XAI Evaluation Paradigms: Plausibility, Faithfulness, and Falsifiability. This dissertation introduces the falsifiability paradigm (third row) as a novel contribution that addresses critical gaps in existing evaluation approaches.}
\label{tab:evaluation_paradigms}
\begin{tabular}{p{2.2cm}p{2.8cm}p{1.3cm}p{1.8cm}p{3cm}p{1.4cm}p{2.2cm}}
\toprule
\textbf{Evaluation Approach} & \textbf{Measures} & \textbf{Testable Prediction?} & \textbf{Ground Truth Required?} & \textbf{Example Metrics} & \textbf{Daubert Standard?} & \textbf{Key Papers} \\
\midrule
\textbf{Plausibility}
& Subjective interpretability, alignment with human intuition
& No
& No
& Human ratings (``looks reasonable''), attention visualization
& Fails
& Attention mechanisms, visualization studies \\
\midrule
\textbf{Faithfulness}
& Correlation with model behavior under perturbation
& No (correlation only)
& Partial (perturbation effects)
& Deletion AUC, insertion AUC, AOPC, infidelity
& Partial
& Adebayo et al. 2018, Kindermans et al. 2019, Tomsett et al. 2020 \\
\midrule
\textbf{Falsifiability} \newline \textit{(This work)}
& Counterfactual predictions: high-attribution features cause large $\Delta$-score, low-attribution features cause small $\Delta$-score
& Yes
& Yes (realized $\Delta$-score from counterfactuals)
& Correlation coefficient $\rho$, p-value, separation margin $\epsilon$
& Passes
& This dissertation (novel framework) \\
\bottomrule
\end{tabular}
\end{table}

% Usage notes:
% - Include in chapter_02_literature_review.tex with: \input{../tables/chapter_02_literature/table_2_2_evaluation_paradigms}
% - Or use full path: \input{tables/chapter_02_literature/table_2_2_evaluation_paradigms}
% - Requires: \usepackage{booktabs} in preamble
% - Landscape orientation not needed (fits on portrait page)

% - Or use full path: % Table 2.2: Evaluation Paradigms Comparison
% Compares plausibility, faithfulness, and falsifiability evaluation approaches
% Location: Chapter 2, Literature Review
% Alt-text: Comparison table showing three XAI evaluation paradigms (Plausibility, Faithfulness, Falsifiability)
% across seven dimensions: evaluation approach, what is measured, testable predictions, ground truth required,
% example metrics, Daubert standard compliance, and key papers

\begin{table}[htbp]
\centering
\small
\caption{Comparison of XAI Evaluation Paradigms: Plausibility, Faithfulness, and Falsifiability. This dissertation introduces the falsifiability paradigm (third row) as a novel contribution that addresses critical gaps in existing evaluation approaches.}
\label{tab:evaluation_paradigms}
\begin{tabular}{p{2.2cm}p{2.8cm}p{1.3cm}p{1.8cm}p{3cm}p{1.4cm}p{2.2cm}}
\toprule
\textbf{Evaluation Approach} & \textbf{Measures} & \textbf{Testable Prediction?} & \textbf{Ground Truth Required?} & \textbf{Example Metrics} & \textbf{Daubert Standard?} & \textbf{Key Papers} \\
\midrule
\textbf{Plausibility}
& Subjective interpretability, alignment with human intuition
& No
& No
& Human ratings (``looks reasonable''), attention visualization
& Fails
& Attention mechanisms, visualization studies \\
\midrule
\textbf{Faithfulness}
& Correlation with model behavior under perturbation
& No (correlation only)
& Partial (perturbation effects)
& Deletion AUC, insertion AUC, AOPC, infidelity
& Partial
& Adebayo et al. 2018, Kindermans et al. 2019, Tomsett et al. 2020 \\
\midrule
\textbf{Falsifiability} \newline \textit{(This work)}
& Counterfactual predictions: high-attribution features cause large $\Delta$-score, low-attribution features cause small $\Delta$-score
& Yes
& Yes (realized $\Delta$-score from counterfactuals)
& Correlation coefficient $\rho$, p-value, separation margin $\epsilon$
& Passes
& This dissertation (novel framework) \\
\bottomrule
\end{tabular}
\end{table}

% Usage notes:
% - Include in chapter_02_literature_review.tex with: \input{../tables/chapter_02_literature/table_2_2_evaluation_paradigms}
% - Or use full path: \input{tables/chapter_02_literature/table_2_2_evaluation_paradigms}
% - Requires: \usepackage{booktabs} in preamble
% - Landscape orientation not needed (fits on portrait page)

% - Requires: \usepackage{booktabs} in preamble
% - Landscape orientation not needed (fits on portrait page)

% - Or use full path: % Table 2.2: Evaluation Paradigms Comparison
% Compares plausibility, faithfulness, and falsifiability evaluation approaches
% Location: Chapter 2, Literature Review
% Alt-text: Comparison table showing three XAI evaluation paradigms (Plausibility, Faithfulness, Falsifiability)
% across seven dimensions: evaluation approach, what is measured, testable predictions, ground truth required,
% example metrics, Daubert standard compliance, and key papers

\begin{table}[htbp]
\centering
\small
\caption{Comparison of XAI Evaluation Paradigms: Plausibility, Faithfulness, and Falsifiability. This dissertation introduces the falsifiability paradigm (third row) as a novel contribution that addresses critical gaps in existing evaluation approaches.}
\label{tab:evaluation_paradigms}
\begin{tabular}{p{2.2cm}p{2.8cm}p{1.3cm}p{1.8cm}p{3cm}p{1.4cm}p{2.2cm}}
\toprule
\textbf{Evaluation Approach} & \textbf{Measures} & \textbf{Testable Prediction?} & \textbf{Ground Truth Required?} & \textbf{Example Metrics} & \textbf{Daubert Standard?} & \textbf{Key Papers} \\
\midrule
\textbf{Plausibility}
& Subjective interpretability, alignment with human intuition
& No
& No
& Human ratings (``looks reasonable''), attention visualization
& Fails
& Attention mechanisms, visualization studies \\
\midrule
\textbf{Faithfulness}
& Correlation with model behavior under perturbation
& No (correlation only)
& Partial (perturbation effects)
& Deletion AUC, insertion AUC, AOPC, infidelity
& Partial
& Adebayo et al. 2018, Kindermans et al. 2019, Tomsett et al. 2020 \\
\midrule
\textbf{Falsifiability} \newline \textit{(This work)}
& Counterfactual predictions: high-attribution features cause large $\Delta$-score, low-attribution features cause small $\Delta$-score
& Yes
& Yes (realized $\Delta$-score from counterfactuals)
& Correlation coefficient $\rho$, p-value, separation margin $\epsilon$
& Passes
& This dissertation (novel framework) \\
\bottomrule
\end{tabular}
\end{table}

% Usage notes:
% - Include in chapter_02_literature_review.tex with: % Table 2.2: Evaluation Paradigms Comparison
% Compares plausibility, faithfulness, and falsifiability evaluation approaches
% Location: Chapter 2, Literature Review
% Alt-text: Comparison table showing three XAI evaluation paradigms (Plausibility, Faithfulness, Falsifiability)
% across seven dimensions: evaluation approach, what is measured, testable predictions, ground truth required,
% example metrics, Daubert standard compliance, and key papers

\begin{table}[htbp]
\centering
\small
\caption{Comparison of XAI Evaluation Paradigms: Plausibility, Faithfulness, and Falsifiability. This dissertation introduces the falsifiability paradigm (third row) as a novel contribution that addresses critical gaps in existing evaluation approaches.}
\label{tab:evaluation_paradigms}
\begin{tabular}{p{2.2cm}p{2.8cm}p{1.3cm}p{1.8cm}p{3cm}p{1.4cm}p{2.2cm}}
\toprule
\textbf{Evaluation Approach} & \textbf{Measures} & \textbf{Testable Prediction?} & \textbf{Ground Truth Required?} & \textbf{Example Metrics} & \textbf{Daubert Standard?} & \textbf{Key Papers} \\
\midrule
\textbf{Plausibility}
& Subjective interpretability, alignment with human intuition
& No
& No
& Human ratings (``looks reasonable''), attention visualization
& Fails
& Attention mechanisms, visualization studies \\
\midrule
\textbf{Faithfulness}
& Correlation with model behavior under perturbation
& No (correlation only)
& Partial (perturbation effects)
& Deletion AUC, insertion AUC, AOPC, infidelity
& Partial
& Adebayo et al. 2018, Kindermans et al. 2019, Tomsett et al. 2020 \\
\midrule
\textbf{Falsifiability} \newline \textit{(This work)}
& Counterfactual predictions: high-attribution features cause large $\Delta$-score, low-attribution features cause small $\Delta$-score
& Yes
& Yes (realized $\Delta$-score from counterfactuals)
& Correlation coefficient $\rho$, p-value, separation margin $\epsilon$
& Passes
& This dissertation (novel framework) \\
\bottomrule
\end{tabular}
\end{table}

% Usage notes:
% - Include in chapter_02_literature_review.tex with: \input{../tables/chapter_02_literature/table_2_2_evaluation_paradigms}
% - Or use full path: \input{tables/chapter_02_literature/table_2_2_evaluation_paradigms}
% - Requires: \usepackage{booktabs} in preamble
% - Landscape orientation not needed (fits on portrait page)

% - Or use full path: % Table 2.2: Evaluation Paradigms Comparison
% Compares plausibility, faithfulness, and falsifiability evaluation approaches
% Location: Chapter 2, Literature Review
% Alt-text: Comparison table showing three XAI evaluation paradigms (Plausibility, Faithfulness, Falsifiability)
% across seven dimensions: evaluation approach, what is measured, testable predictions, ground truth required,
% example metrics, Daubert standard compliance, and key papers

\begin{table}[htbp]
\centering
\small
\caption{Comparison of XAI Evaluation Paradigms: Plausibility, Faithfulness, and Falsifiability. This dissertation introduces the falsifiability paradigm (third row) as a novel contribution that addresses critical gaps in existing evaluation approaches.}
\label{tab:evaluation_paradigms}
\begin{tabular}{p{2.2cm}p{2.8cm}p{1.3cm}p{1.8cm}p{3cm}p{1.4cm}p{2.2cm}}
\toprule
\textbf{Evaluation Approach} & \textbf{Measures} & \textbf{Testable Prediction?} & \textbf{Ground Truth Required?} & \textbf{Example Metrics} & \textbf{Daubert Standard?} & \textbf{Key Papers} \\
\midrule
\textbf{Plausibility}
& Subjective interpretability, alignment with human intuition
& No
& No
& Human ratings (``looks reasonable''), attention visualization
& Fails
& Attention mechanisms, visualization studies \\
\midrule
\textbf{Faithfulness}
& Correlation with model behavior under perturbation
& No (correlation only)
& Partial (perturbation effects)
& Deletion AUC, insertion AUC, AOPC, infidelity
& Partial
& Adebayo et al. 2018, Kindermans et al. 2019, Tomsett et al. 2020 \\
\midrule
\textbf{Falsifiability} \newline \textit{(This work)}
& Counterfactual predictions: high-attribution features cause large $\Delta$-score, low-attribution features cause small $\Delta$-score
& Yes
& Yes (realized $\Delta$-score from counterfactuals)
& Correlation coefficient $\rho$, p-value, separation margin $\epsilon$
& Passes
& This dissertation (novel framework) \\
\bottomrule
\end{tabular}
\end{table}

% Usage notes:
% - Include in chapter_02_literature_review.tex with: \input{../tables/chapter_02_literature/table_2_2_evaluation_paradigms}
% - Or use full path: \input{tables/chapter_02_literature/table_2_2_evaluation_paradigms}
% - Requires: \usepackage{booktabs} in preamble
% - Landscape orientation not needed (fits on portrait page)

% - Requires: \usepackage{booktabs} in preamble
% - Landscape orientation not needed (fits on portrait page)

% - Requires: \usepackage{booktabs} in preamble
% - Landscape orientation not needed (fits on portrait page)

% - Requires: \usepackage{booktabs} in preamble
% - Landscape orientation not needed (fits on portrait page)
